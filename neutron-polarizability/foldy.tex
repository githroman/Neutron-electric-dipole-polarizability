In order to extract the lattice measurement of the neutron's electric polarizability,
one has to calculate the point-like contribution present in its energy spectrum. Such a 
a contribution to dipole electric polarizability in long-know~\cite{Foldy:1959zza}. There are ten
leading Foldy contributions for a zero-momentum neutron, which were determined
from a Foldy-Wouthuysen transformation constructed in such a form as to yield
the energy shift of a neutron in the presence of an external electromagnetic field.
These contributions are reported in~\cite{Saenz:2020yxy}, and the Foldy-Wouthuysen
electric polarizability contribution $\alpha_{\text{FW}}$ is
\begin{equation}
\alpha_{\text{FW}}=\frac{-\mu^2}{m_n}
\label{fwcontrib}
\end{equation}
where $\mu$ is the anomalous magnetic moment of the neutron and $m_n$ is the 
neutron mass. Given that the lattice measurements of the electric polarizability are
conducted at $m_\pi=$ 357 MeV, the point-like contribution in \eq{fwcontrib} has to be
evaluated at the pion mass scale. In~\cite{Green:2019zhh}, the isovector and
isoscalar normalized anomalous magnetic moments are given by 
$\kappa_v^{\text{norm}}=2.518(57)$ and $\kappa_s^{\text{norm}}=-0.030(22)$
in magnetons\footnote{Note, in this work a pion mass of $m_\pi=$ 355 MeV was given, which however corresponds to the same ensemble as the present one.}. 
The neutron anomalous magnetic moment is given by
\begin{equation}
\kappa=\frac{\kappa_s^{\text{norm}}-\kappa_v^{\text{norm}}}{2}=-1.274(31)
\end{equation}
As a function of the neutron mass $m_n$, the magnetic moment in GeV$^{-1}$ is
\begin{equation}
\mu=\kappa\sqrt{\frac{1}{137}}\frac{1}{2m_n^{\text{phys}}}
\label{anomalous}
\end{equation}
Evaluating~\ref{anomalous} at the physical neutron mass $m_n^{\text{phys}}$, as given 
in~\cite{ParticleDataGroup:2016lqr}, the magnetic moment is then
\begin{equation}
\mu_{357\text{ MeV}}=-0.0579(14)\text{ GeV}^{-1}
\label{magmom}
\end{equation}
On the other hand, at $m_\pi=355.98(80)$ MeV the {\cred neutron mass is $m_n=1154.8(80)$???} 
MeV, as given in~\cite{LHPC:2010jcs}. This mass value, along with the result given in~\ref{magmom}, lead to 
a Foldy-Wouthuysen contribution (c.f.~\ref{fwcontrib})
\begin{equation}
\alpha_{\text{FW}}^{357\text{ MeV}}=(-0.2232\pm7.7\times 10^{-7})\text{ }10^{-4}\text{ fm}^{-3} {\cred error???}
\end{equation}
As detailed in~\cite{Saenz:2020yxy},~\eq{fwcontrib} results from the energy shift of a 
poin-tlike neutron, so it must be subtracted from the $\alpha_E$ obtained from the lattice 
measurements described in this work.
%%%%%%%%%%%%%%%%%%%%%%%%%%%
