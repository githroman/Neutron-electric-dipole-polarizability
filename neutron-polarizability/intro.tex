In~\cite{Babusci:1998ww}, the general structure of the cross section in Compton scattering 
with polarized photons and nucleons is discussed. A low-energy expansion
of the scattering amplitudes $T_{if}$ leads to a cross section expression involving
ten nucleon structure parameters, i.e., dipole, quadrupole, dispersion, and
spin polarizabilities. In addition, $|T_{if}|^2$, are written in terms of invariant
amplitudes $T_i$, which are decomposed into Born and non-Born contributions.
Specifically, the spin-independent part of the non-Born contribution to the scattering amplitudes
involve the coefficients $\alpha_E$ and $\beta_M$, i.e., respectively, the
dipole electric and magnetic polarizabilities of the nucleon, which describe
the magnitude of the electric and magnetic dipole moments induced on the nucleon by an
applied electric field $\overrightarrow E$ or magnetic field $\overrightarrow B$\cite{Hagelstein:2020vog}.

 As detailed in~\cite{Detmold:2006vu},the polarizabilities measure the response of the 
 nucleon quark constituents to an applied electromagnetic field. The effective Hamiltonian 
 of the interaction of a nucleon with an external electromagnetic field
is written in terms of ten leading-order terms (in Gaussian units):
\footnote{Polarizability units: In the SI system, there is an additional prefactor of 4$\pi$ and the permittivity $\varepsilon_0$ and permeability $\mu_0$ of the vacuum enter in the corresponding terms in the Hamiltonian.}
\begin{equation}
\begin{split}
H_{\text{eff}^{(2)}}=&-\frac{1}{2}(\alpha_E E^2+\beta_M B^2
+\gamma_{E1}\sigma\cdot\left(E\times \dot E\right)\\
&+\gamma_{M1}\sigma\cdot \left(B\times \dot B\right)
-2\gamma_{E2}E_{ij}\sigma_iB_j+2\gamma_{M2}B_{ij}\sigma_iE_j\\
&+\alpha_{E\nu}\dot E^2+\beta_{M\nu}\dot B^2
+\frac{1}{6}\alpha_{E2}E_{ij}^2+\frac{1}{6}\beta_{M2}B_{ij}^2+\dots)
\end{split}
\label{Heffective}
\end{equation}
where the quadrupole strengths of the electric and magnetic fields are
\begin{equation}
\begin{split}
E_{ij}=&\frac{1}{2}\left(\nabla_iE_j+\nabla_jE_i\right)\\ 
B_{ij}=&\frac{1}{2}\left(\nabla_iB_j+\nabla_jB_i\right)
\end{split}
\end{equation}
and $\alpha_E$ and $\beta_M$ are the electric and magnetic
polarizabilities, $\gamma_{E1}$, $\gamma_{M1}$, $\gamma_{E2}$, and $\gamma_{M2}$ 
are the spin polarizabilities, $\alpha_{E\nu}$, $\beta_{M\nu}$ are the dispersion polarizabilities,
and $\alpha_{E2}$ and $\beta_{M2}$ are the quadrupole polarizabilities~\cite{Babusci:1998ww, Levchuk:1999zy}.
By writing the effective Hamiltonian in the form of~\eq{Heffective}, one assumes a sufficiently
localized nucleon wave function in the sense that its energy depends on the local values of the
electric and magnetic fields and  their derivatives. This matter and the inherent limitations
of~\eq{Heffective} are discussed in more detail in~\cite{Saenz:2020yxy}.

Tripartite efforts are currently underway to determine the nucleon polarizabilities
by experiments, analytically in Chiral Perturbation Theory ($\chi$PT), and by numerical calculation in lattice QCD.

In experimental results, the sum of the electric and magnetic polarizabilities is
an integral of the total photoabsorption cross section in Compton scattering,
i.e., the Baldin sum rule. Additional sum rules and recommended nucleon polarizability values
based on experimental results are presented in~\cite{Schumacher:2019ikn}.
The polarizability addends in the Baldin sum can be singled out by measuring the angular
distribution of the photoabsorption in low-energy Compton scattering~\cite{Hagelstein:2020vog,Levchuk:1999zy}.
For example, in~\cite{Levchuk:1999zy}, deuteron Compton scattering experimental data were fit to
extract the electric and magnetic polarizability of the neutron. An alternative experimental method
is presented in~\cite{Schumacher:2019ikn}, the electric polarizability is measured in the scattering of slow
neutrons in the presence of the electric field of a heavy nucleus, and results for the neutron
polarizabilities are presented as the average of the individual results from deuteron Compton
scattering and neutron electromagnetic scattering by the electric field of a lead nucleus.
Phenomenological extraction of the dipole electric and magnetic, and spin polarizabilities are discussed in~\cite{Holstein:1999uu, Drechsel:2002ar, Hildebrandt:2003fm, Schumacher:2005an, Pasquini:2007hf, Pasquini:2010zr, Griesshammer:2012we, McGovern:2012ew, Holstein:2013kia, COMPTONMAX-lab:2014cve, A2:2014iky, Gryniuk:2015eza, Gryniuk:2016gnm, Hagelstein:2015egb, Griesshammer:2017txw, Pasquini:2017ehj, Pasquini:2018wbl, Pasquini:2019nnx, Miskimen:2019kwu, Martel:2019tgp, A2:2019bqm, Melendez:2020ikd}. Dispersion relation analyses for experimental data of Compton scattering off protons or deuteron at low-energy 
(below pion production or near $\Delta(1232)$ resonance, for example) are discussed to extract nucleon
structure constants such as the polarizabilities.

 Chiral Perturbation Theory, a low-energy effective field theory of the
strong interaction, provides a tool to explore low-energy hadronic physics, such as Compton
Scattering off the nucleon, leading to the calculation of the nucleon electromagnetic polarizabilities
in chiral theory~\cite{Hagelstein:2020vog, Bernard:1991rq, Bernard:1991ru}.

Calculating hadron mass shifts in the presence of external electromagnetic fields in lattice QCD paves a way to calculate nucleon polarizabilities. Some of these efforts are recounted in \cite{Fiebig:1988en, Christensen:2004ca, Lee:2005dq, Shintani:2006xr, Engelhardt:2007ub, Engelhardt:2009ryp, Detmold:2009dx, Detmold:2010ts, Engelhardt:2011qq, Alexandru:2009id, Lee:2010dq, Lee:2011gz, Lujan:2014kia, Freeman:2014kka, Luschevskaya:2014lga, Lujan:2016ffj, Primer:2013pva, Bignell:2018acn, Bignell:2020xkf, Bignell:2020dze, He:2020ysm, Bignell:2020aye}.
Measurements on the lattice are performed on finite volumes and at unphysical (heavy) pion masses. The polarizabilities are given by the neutron mass shift in the presence of external static electric and 
magnetic fields, precisely on the part of the mass shift that is quadratic-in-the-fields. Chiral effective
theory aims to connect lattice results to the pion mass and infinite volume physical limits~\cite{Detmold:2006vu, He:2020ysm, Hildebrandt:2003fm, Lensky:2015awa, Griesshammer:2015ahu}. As mentioned, the polarizability coefficients in~\eq{Heffective} can be 
connected to the Compton scattering amplitudes in the low-energy limits. Matching the background field
calculations of the polarizabilities in lattice QCD to the effective field theory description of the Compton
amplitudes is explored amply in~\cite{Lee:2013lxa, Lee:2014iha}, for example.

In this work, we focus on the lattice measurements of the electric polarizability $\alpha_E$ of the neutron.
While other works on lattice hadron polarizability measurements have been performed in the quenched approximation~\cite{Fiebig:1988en, Wilcox:1996vx, Wilcox:1997ee, Zhou:2002km, Christensen:2002wh, Christensen:2004ca, Lee:2005dq, Lee:2005vv}, in this work, we use a dynamical quark ensemble with a pion mass of 357 MeV, and improve the measurements of the electric polarizability of the neutron as compared to those reported in~\cite{Engelhardt:2007ub, Engelhardt:2009ryp} by considering connected and disconnected diagram contributions. In future work, we will present the lattice measurement of the spin electric polarizability $\gamma_{E1}$.

