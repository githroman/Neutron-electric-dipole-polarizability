
The numerical results discussed in this work are the result of the analysis of 448 gauge configurations generated with dynamical asqtad staggered quarks~\cite{Lepage:1998vj}, provided by the MILC collaboration~\cite{Bernard:2001av}. The lattice spacing $a =$ 0.124 fm, with bare quark
masses $am_l$ = 0.01 and $am_s$ = 0.05, which corresponds to a pion mass of 357 MeV. 
The quark lines of the diagrams in \fig{fig:diagrams} that contribute to the neutron electric polarizability were
populated with domain wall quarks, the disconnected contributions were estimated using complex $Z(2)$ stochastic sources. A technicality has been identified in~\cite{Engelhardt:2007ub}: a time-independent Hamiltonian ensures a stationary neutron wave function, while a time dependence is introduced by smeared neutron sinks, which would have to be addressed by considering additional contributions from additional diagrams to those shown in \fig{fig:diagrams}. In order to avoid these systematic effects we use point-like neutron sinks  which are time-independent and invariant under gauge transformations of the external field. After first measurements we found the denominator of the correlator ratio $R_2(t)$ in \eq{ratios} to be quite noisy, and therefore improved its statistics. For the results including disconnected diagrams in particular, the improved statistics however did not pay off and therefore we also quote the results of the original measurements.  

\todo{describe electric field and 1 vs. 3 window fits...

A constant electric field is introduced in the 3-direction of the gauge field of the form
\begin{equation}
A_3=E(t-t_0).
\label{gauge}
\end{equation}
Different choices of time $t_0$ correspond to time shifts of the gauge component. Measurements at different $t_0$ allow one to treat
the finite lattice effects, as the neutron energy spectrum depends on the gauge field $A$.
It has been shown that there is a strong dependence of the nucleon polarizabilities on
lattice volume, and it is claimed that a correction below 10\% can be made. The finite volume corrections are
found by studying the volume dependence of the values of polarizability obtained in calculations performed
on lattices of different size~\cite{Lujan:2016ffj}.
The constant gauge field dominant effect is further elaborated in~\cite{Engelhardt:2007ub}. Furthermore, 
the present work used only one spatial lattice volume. Thus, to treat finite-size effects, polarizability
measurements were performed at three different time values corresponding to three constant shifts of $A_3$.
The polarizability measurements where made at $t_0=-10a,\ 0,\ 6a$. These choices correspond 
to the three-window analysis, which probes the behavior of the correlation ratio $R_2$ at three times, 
thus defining the parabola of temporal slopes of the correlator ratio. In turn, the extremal slope
corresponds, with a minus sign, to the mass shift of the neutron at the stationary point. In our work,
the extremal slope was corrected for curvature effects; the need for such a correction arises
from our fitting of linear functions, for a selected range of data points, to the cubic behavior
of the correlator ratio, as will be discussed shortly.

The generic behavior of the correlator ratios in \eq{ratios} is cubic as a function of time steps $t/a$.
In our 3-window analysis, three values of the slope are obtained from a selected range of different 
data sets. These three slope values correspond to three points in a quadratic function, a parabola that
results from the derivative of the cubic function that represents the generic behavior of the correlator
ratios. Note that in each time window, a range of data has to be selected to obtain from it the slope.
In our work, linear functions were fitted to the ranges of each of the data sets. This procedure yields
slope, or points in a parabola, whose values deviate from the true values obtained from a
cubic function. To define the correction to the minima, we defined a range $M$ of data points,
which is the difference between the final and initial time steps in the range. Assuming that a cubic
function describes a set of perfect data points, we considered the sum of the squares of said
perfect data point in a cubic function. By minimizing the sum, in turn, with respect to the slope
$p$ of the linear term and with respect to the constant term $q$ of the linear function, an expression
for $p$ in terms of $M$ and of the coefficients of the cubic function was obtained. The form of
$p$ is quadratic in time step $s$, say. The three values on the parabola
were taken as the values of the slope at $s$= 21,11, 5. On the other hand, the exact minimum
of a parabola is obtained in a straightforward calculation, starting from a function quadratic in time 
step $t$. By comparing the forms of the quadratic in $s$ and quadratic in $t$ functions, a correction
$\Delta p$ for the minimum of the parabola was found. Said correction has an interesting and useful
feature; it is a constant that depends on the fitting range $M$. This means that the same correction 
may be applied to any and all time windows. The correction of the extremum was applied 
to the extracted minima of the fitted quadratic function.

The slopes presented in this work were obtained by performing $\chi^2$ 
fits to the $R_2(t)$ data for a range of choice $5a\le t\le7a$ in the three-window analysis,
in which linear functions are fitted for the selected range at the three time windows
of the cubic correlator ratio. As mentioned, the slopes obtained from these fits are three points in
a quadratic function, which represents the derivative of the cubic function, for which the
extremal value is determined. These extremal values were corrected for curvature of the
fitted function, as justified above. This work also considers the fitting ranges $4a\le t\le 8a$ and $3a\le t\le 9a$ 
as alternatives. 
}
