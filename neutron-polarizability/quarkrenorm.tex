As discussed already in~\cite{Engelhardt:2007ub, Engelhardt:2009ryp}, we need to renormalize our results. The renormalization factor $z_v=3/Q$, where $Q$ is a measurement of the number of valence quarks in the neutron, is
deduced by measuring the three-point function without weighting by the quark electric charge $q_f$ in the presence of an external gauge field of the form 
\begin{equation}
A_o=\delta(x_0-t)
\end{equation}
for some time $t$ between the hadron source and sink. The lattice regularization makes
the number $Q=\int d^3xj_o$ deviate from 3, where $j_0$ is the temporal component of the 
quark current. One can obtain the number of quarks by averaging
measurements of $t$ at different lattice times. In this work, these measurements
where taken at insertion times $t$ in the interval $4a\le t \le8a$, while 
the sink is introduced at $t_{\text{sink}}=13a$. The uncertainties in both $Q$ and
$z_v=3/Q$ were obtained by the jackknife method. As remarked in~\cite{Engelhardt:2009ryp},
light and strange quark lattice currents have different renormalization factors.
To measure the aforementioned renormalization factors, the ratios $Q_l, Q_s$,
respectively light and strange quarks, of the data with
and without insertion should reach plateau values. The renormalization factors are then
$3/Q_l$ and $3/Q_s$. Taking into account all diagrams in \fig{fig:diagrams}, every connected insertion is 
renormalized by the first value, but either renormalizes every loop insertion, e.g.,
a connected plus strange loop, or a mixed light-strange loop, is renormalized by a factor $9(Q_lQ_s)$.
The charge renormalization factors for light and strange quarks on the $m_{\pi}=357$ MeV ensemble were measured to be
\begin{align}
Q_l=2.629(20)&\Rightarrow z_{v,l}=1.1412(86)\\
Q_s=2.7354(22)&\Rightarrow z_{v,s}=1.09671(89)
\end{align}